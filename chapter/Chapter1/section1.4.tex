\section{Cấu trúc tài liệu}
Để giúp người đọc dễ nắm bắt được nội dung tài liệu, chúng tôi đã tổ chức tài liệu này thành 7 chương. Trong phần này, chúng tôi sẽ giới thiệu sơ lược về
từng chương trong tài liệu.

7 chương trong tài liệu bao gồm:
\begin{itemize}
    \item Chương 1: Giới thiệu \ref{chapter:introduction}
    \item Chương 2: Kiến thức nền tảng \ref{chapter:preliminaries}
    \item Chương 3: Thu thập và phân tích yêu cầu \ref{chapter:requirements}
    \item Chương 4: Phân tích, thiết kế hệ thống \ref{chapter:architecture}
    \item Chương 5: Quá trình xây dựng AI \ref{chapter:ai}
    \item Chương 6: Cài đặt và kiểm thử \ref{chapter:testing}
    \item Chương 7: Tổng kết và định hướng tương lai \ref{chapter:conclusion}
\end{itemize}

Mỗi chương sẽ tập trung vào một khía cạnh cụ thể của dự án, được ghi lại và tổng hợp trong suốt quá trình làm việc theo quy trình phát triển phần mềm Scrum của
nhóm chúng tôi. Những kiến thức nền tảng cần có để đọc và theo dõi dự án được trình bày tại chương 2.
Trong các chương 3 và 4, dự án được phân tích từ những yêu cầu nền tảng được đề ra, sau đó thiết kế kiến trúc hệ thống để đáp ứng các yêu cầu này. Đến chương 4,
chúng ta có thể đã có một hệ thống nghe nhạc cơ bản. Tiếp theo, trong chương 5, chúng tôi trình bày chi tiết quá trình xây dựng hệ thống AI tích hợp trong dự án.
Chương 6 tập trung vào quá trình cài đặt và kiểm thử để đảm bảo hệ thống hoạt động ổn định và đáp ứng các yêu cầu đã đề ra. Việc này được thực hiện xuyên suốt
dự án nhưng để thuận tiện cho người đọc, chúng tôi sẽ tổng hợp toàn bộ ca kiểm thử trong chương này. Cuối cùng, chương 7 sẽ tổng kết lại toàn bộ dự án, những đóng
góp chính và định hướng phát triển trong tương lai.