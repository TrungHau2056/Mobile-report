\chapter{Kiến thức nền tảng}

\subsection{Điểm mạnh}
\begin{itemize}
    \item Tính mới cao: Đề cập đến nhiều công nghệ mới và hiện đại như YOLOv10, GraphRAG,\dots
    \item Có lập luận rõ ràng về lý do tại sao chọn các công nghệ này
    \item Cấu trúc mạch lạc, logic.
    \item Các kiến thức nền tảng gắn kết thực tế để người đọc nắm được các chức năng chính.
\end{itemize}

\subsection{Điểm hạn chế}
\begin{itemize}
    \item Thiếu sót thành phần trong cấu trúc báo cáo khi ở phần tổng quan có nhắc đến FastAPI và Neo4j những chưa thấy giải thích kĩ về hai thành phần này trong phần kiến thức nền tảng.
    \item Mức độ chi tiết chưa đồng đều khi các phần khác được phân tích khá kĩ lưỡng nhưng về phần LangChain viết hơi sơ sài chưa giải thích rõ ràng tại sao lại nên dùng LangChain.
\end{itemize}
\subsection{Chưa giải thích kiến trúc MVVM}
Hình \ref{fig:mvvm} cho thấy nhóm đã nói về kiến trúc giao diện theo kiểu MVVM nhưng lại chưa chỉ rõ kiến trúc này gồm những thành phần nào và hoạt động như thế nào.
\begin{figure}[H]
    \includegraphics[width=\linewidth,height=0.7\textheight,keepaspectratio]{figures/hoang_1.png}
    \centering
    \caption{}
    \label{fig:mvvm}
\end{figure}

