\section{Người phản biện: Đoàn Minh Hoàng}

\vspace{2cm}
Về \textbf{điểm mạnh (Strengths)}:
\begin{itemize}
    \item Có ý tưởng tốt, nội dung được đầu tư rõ ràng
    \item Bài toán có tính thực tế và ứng dụng cao trong đời sống
    \item Giao diện được thiết kế tốt, có độ hoàn thiện cao
\end{itemize}

\vspace{2cm}
Về \textbf{điểm yếu (Weaknesses)}:
\begin{itemize}
    \item Chưa có kiến trúc front-end riêng
    \item Hình 5.2 phần giao diện chưa có giải thích
    \item Hình 5.20 phần giao diện chưa chứng tỏ được số điện nước sẽ bị trừ đi so với tháng trước.
\end{itemize}


\vspace{2cm}
Tôi xin được \textbf{chấm điểm} đề tài như sau:
\begin{itemize}
  \item Về \textbf{chất lượng (quality)}: Độ hoàn thiện tốt, giao diện người dùng được thiết kế tốt, đơn giản và dễ làm quen với người dùng
\textbf{9/10}
  \item Về \textbf{tính rõ ràng (clarity)}: Kiến trúc front end chưa được mô tả riêng, phần kiểm thử có những phần có kết quả giống với kì vọng nhưng lại bị kết luận là failed.
\textbf{8.5/10}
  \item Về \textbf{độ ảnh hưởng (significance)}: Bài toán thực tế, có tính ứng dụng cao.
\textbf{8.5/10}
  \item Về \textbf{độ độc đáo (originality)}: OCR điện nước tiện lợi
\textbf{9/10}
\end{itemize}
Đi kèm với những \textbf{câu hỏi} sau:
\begin{itemize}
  \item Front-end quản lí các trạng thái bằng cách nào?
  \item Việc chụp ảnh CCCD có thể làm người dùng bị lộ dữ liệu không? Nếu có thì giải quyết như thế nào?
\end{itemize}

Đánh giá cuối cùng: Nhìn chung đây là một giải pháp có ý tưởng hay, được đầu tư tốt, thiết kế giao diện tốt, có sử dụng AI và việc đọc số điện nước bằng scan là mới mẻ. Tuy nhiên còn một số lỗi trong báo cáo đã nêu ở trên.

Điểm: \textbf{8.75/10}
