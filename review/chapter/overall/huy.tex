\section{Người phản biện: Nguyễn Quốc Huy}
Về tổng quan(Summary): Đóng góp chính của dự án trong giải quyết vấn đề thực tế:
\begin{itemize}
  \item Tự động hóa quy trình báo cáo thanh toán hàng tháng (scan, lập hóa đơn tự động)
  \item Tạo sản phẩm kết nối chủ nhà người thuê (dành cho chủ nhà và dành cho khách) AI của dự án: Triển khai YOLO để quét điện nước, 
  Chatbot và knowledge graph để trả lời câu hỏi về luật
\end{itemize}

\vspace{2cm}
Về \textbf{điểm mạnh (Strengths)}:
\begin{itemize}
  \item Về động lực: Tìm ra gap của ứng dụng hiện tại từ đó các yêu cầu đưa ra bám sát xung quanh hai vấn đề thực tiễn.
  \item Kiến thức nền tảng: trình bày khá rõ ràng, đầy đủ
  \item Phân tích đặc tả yêu cầu: Các yêu cầu chức năng và phi chức năng đầy đủ cũng được thể hiện ở phần kiểm thử.
  \item AI: Triển khai tích hợp thành công công nghệ của LLM và CV vào hệ thống, bám sát nhu cầu thực tiễn, giúp cải thiện trải nghiệm người dùng
  \item Tính ứng dụng: Có tính ứng dụng cao với hiện trạng nhà ở tại Việt Nam, đã có người dùng sử dụng thực tế
  \item Độ hoàn thiện: Nhóm đã có sự đầu tư vào sản phẩm, có tiến trình phát triển trên Github với nhiều tính năng.
  \item UI: tốt, tuân thủ nguyên tắc phát triển.
\end{itemize}

\vspace{2cm}
Về \textbf{điểm yếu (Weaknesses)}:
\begin{itemize}
  \item Một số thuật ngữ được sử dụng không đi kèm giải thích, fail ở những test case có đầu ra kỳ vọng và thực tế giống nhau (trừ điểm clarity)
\item Cơ sở dữ liệu có độ phức tạp nhưng cấu trúc firestone là NoSQL, chi phí cho việc đọc ghi sẽ tăng lên rất nhanh
\item Ứng dụng khi xây dựng server nhanh nhưng phục thuộc nền tảng Firebase, rất khó để mở rộng, và khi triển khai thực tế với lượng người dùng tăng lên là không thực tế.
\item Thiết kế hệ thống vi phạm Client-Server. (1) Client hiện đang là điểm trung chuyển cả 3 dịch vụ khác. (2) Kết nối trực tiếp từ client tới các service sẽ làm tăng vấn đề bảo mật. (3)  Vấn đề debug và update sản phẩm, nếu các ứng dụng ngoài thay đổi API sẽ phải cập nhật ứng dụng, logging không tập trung.
\item KG thể hiện sự phụ thuộc theo cấu trúc mục, chương,... của văn bản luật nhưng về các tác nhân của luật thì chưa.
\item Cơ chế tự động là để người thuê quét, xem xét về tính minh bạch cũng như độ chính xác, chưa khả thi trong thực tế
  
\end{itemize}

Kết luận về điểm yếu: Thiết kế cấu trúc của hệ thống bất cập, một bản thực thi của ý tưởng hoàn thiện nhưng tính khả thi thấp.

\vspace{2cm}
Tôi xin được \textbf{chấm điểm} đề tài như sau:
\begin{itemize}
  \item Về \textbf{chất lượng (quality)}: Độ hoàn thiện của báo cáo là cao. Việc thiết kế các ca sử dụng chỉn chu, các ca kiểm thử được thiết kế tương xứng. 
  Giao diện thiết kế tốt, tuân thủ nguyên tắc
  YOLO và KG đều được xây dựng ổn, tuy nhiên chưa có độ chính xác cao.
\textbf{9/10}
  \item Về \textbf{tính rõ ràng (clarity)}: Còn những lỗi chính tả trong phiên bản review; biểu đồ hoạt động không rõ ràng, 
  phần kiểm thử có những case fail nhưng kỳ vọng thực tế giống nhau. Cấu trúc các phần rõ ràng
\textbf{8/10}
  \item Về \textbf{độ ảnh hưởng (significance)}: Bài toán thực tế, động lực rõ ràng. 
Giải pháp chưa có tính ứng dụng, ổn về mặt ý tưởng. 
Chatbot khó ứng dụng được, thiết kế hệ thống và đánh giá đều cần cải tiến.
\textbf{8/10}
  \item Về \textbf{độ độc đáo (originality)}: Auto billing và scan không mới nhưng tích hợp vào một ứng dụng nhà trọ tại Việt Nam là độc đáo.
\textbf{9/10}
\end{itemize}
Đi kèm với những \textbf{câu hỏi} sau:
\begin{itemize}
  \item Có thể thiết kế Knowledge Graph có thêm phụ thuộc giữa các tác nhân trong luật được không?
  \item Phần thiết kế hệ thống đơn giản nhất có thể làm một server side theo monolithic architecture để wrap lại hết cả 3 dịch vụ trong thiết kế. Tại sao lại để client gọi trực tiếp cả 3 như vậy?
  \item YOLO ra F1 0.57 cho ứng dụng có vẻ không cao? Đây là vấn đề dữ liệu hay giới hạn của mô hình?
  \item Faithful cho một ứng dụng liên quan tới luật là khá thấp. Kể cả thấp, nếu thêm một ablation study về việc không sử dụng Graph với có sẽ khá ấn tượng. Điều này có khả thi?
\end{itemize}

Tôi xin được đánh giá số điểm \textbf{8.5/10} với đánh giá cuối cùng như sau:
Nói chung, hệ thống khá ấn tượng. Về thiết kế hệ thống và khâu kiểm thử cần cải thiện thêm. UI của nhóm rất tốt. Dự án trên Github được làm chỉn chu.
Ý tưởng hay, phân tích đặc tả yêu cầu tốt, tích hợp AI có kiểm thử, tính ứng dụng chưa cao.